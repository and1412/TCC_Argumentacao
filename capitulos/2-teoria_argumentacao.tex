\chapter{Teoria da Argumentação}

O software é um componente de um sistema computacional de interesse, uma vez que vários aspectos relacionados a ele ultrapassam questões técnicas \cite{meirelles2013metrics}, como por exemplo: 

\begin{itemize}
\item O processo de desenvolvimento de software;
\item Os mecanismos econômicos (gerenciais, competitivos, sociais, cognitivos, etc.) que regem esse desenvolvimento e seu uso;
\item O relacionamento entre desenvolvedores, fornecedores e usuários de software;
\item Os aspectos éticos e legais relacionados ao software;
\end{itemize}

%Diferenciar software livre de software proprietario
O entendimento desses quatro pontos é o que diferencia softwares ditos restritos dos livres, bem como, é o que define o que é conhecido como ``ecossitema do software livre''. O princípio básico desse ecossistema é promover a liberdade do usuário, sem discriminar quem tem permissão para usar um software e seus limites de uso, baseado na colaboração e num processo de desenvolvimento aberto \cite{meirelles2013metrics}.

Ao contrário do software restrito, o software livre tem a característica do compartilhamento do seu código-fonte. Essa característica oferece vantagens em relação ao software proprietário. O compartilhamento permite a simplificação de aplicações personalizadas, já que não necessitam serem codificadas do zero, podendo se basear em soluções existentes. Outra vantagem é a melhoria da qualidade \cite{raymond1999cathedral}, por conta da grande quantidade de colaboradores, que com diferentes perspectivas e necessidades, propõem melhorias para o sistema, além de identificar e corrgir bugs com mais rapidez.

Em resumo, software livre representa uma classe de sistemas de software, os quais são distribuídos sob licenças cujos termos permitem aos seus usuários utilizar, estudar, modificar e redistribuir o software \cite{terceiro2012freesoftware}. 

\section{Processo de Desenvolvimento de Software Livre}
\label{sec-proc-sl}

O aspecto mais importante de um software livre, sob a perspectiva da Engenharia de Software é o seu processo de desenvolvimento. Um projeto de software livre começa quando um desenvolvedor individual ou uma organização decidem tornar um projeto de software acessível ao público. Seu código-fonte é licenciado de forma a permitir seu acesso e alterações subsequentes por qualquer pessoa. Tipicamente, qualquer pessoa pode contribuir com o desenvolvimento, mas mantenedores ou líderes decidem quais contribuições serão incorporadas à release oficial. Não é uma regra, mas projetos de software livre, muitas vezes, recebem colaboração de pessoas geograficamente distantes que se organizam ao redor de um ou mais líderes \cite{corbucci2011freemethods}. 

Há características presentes no software livre que, a princípio, tornam incompatível a aplicação de métodos ágeis em seu desenvolvimento, por exemplo. Entre essas características estão a distância entre os desenvolvedores e a diversidade entre suas culturas, que dificultam a comunicação, um dos principais valores dos métodos ágeis. Entretanto, o sucesso resultante de alguns projetos de software livre, como é o caso do Kernel do Linux \footnote{\url{https://www.kernel.org/}}, fizeram surgir estudos com foco na união dessas duas vertentes.

Analisando um pouco melhor projetos de software livre, é possível notar que esses compartilham princípios e valores presentes no manifesto ágil \footnote{\url{http://agilemanifesto.org/}}. Adaptação a mudanças, trabalhar com \textit{feedback} contínuo, entregar funcionalidades reais, respeitar colaboradores e usuários e enfrentar desafios, são qualidades esperadas em desenvolvedores que utilizam métodos ágeis e são naturalmente encontradas em projetos de software livre.

Num trabalho realizado, \citeonline{corbucci2011freemethods} analisa semelhanças entre projetos de software livre e métodos ágeis, através de uma relação entre os quatro valores enunciados no manifesto ágil e práticas realizadas em projetos livres. 
%
%TODO: descrever relação entre valores do manifesto ágil e práticas de projetos de software livre.
% Não precisa colocar o manifesto ágil vai direto para a relação.
% TODO: Não explicou a questão dos níveis de colaboração, desde o usuário passivo até um desenvolvedor core, exemplificando que, no momento, através deste trabalho, você é um desenvolvedor periférico.
%TODO: Reescrever com suas palavras no TCC 2 (TESE DO PAULO ABAIXO)
%
Conceitualmente, os valores semelhantes são:

\begin{itemize}

\item {Indivíduos e interações são mais importantes que processos e ferramentas.}

\item {Software em funcionamento é mais importante que documentação abrangente.}

\item {Colaboração com o cliente (usuários) é mais importante que negociação de contratos.}

\item {Responder às mudanças é mais importante que seguir um plano.}

\end{itemize}

Além disso, várias práticas disseminadas pelas metodologias ágeis são usadas no
dia-a-dia dos desenvolvedores e equipes das comunidades
de software livre~\cite{corbucci2011freemethods}:

\begin{itemize}

\item {Código compartilhado (coletivo);}
\item {Projeto simples;}
\item {Repositório único de código;}
\item {Integração contínua;}
\item {Código e teste;}
\item {Desenvolvimento dirigido por testes, e}
\item {Refatoração.}

\end{itemize}

Observar e entender esses aspectos nos projetos de software livre tornam-se
relevantes à medida que muitos projetos de software livre não vão além dos
estágios iniciais e muitos acabam sendo abandonados antes de produzir
resultados razoáveis.
%
Isso sugere que, mesmo com o sucesso de alguns projetos de software livre,
as comunidades, com ou sem a participação de empresas, podem avançar no
acompanhamento do desenvolvimento dos projetos de software livre que participam.
%
Olhar o processo de desenvolvimento de software livre do ponto de vista da 
Engenharia de Software e as possíveis sinergias com os métodos ágeis podem
contribuir para um melhor rendimento dessa disposição na criação e colaboração
em torno de projetos de software livre~\cite{meirelles2013metrics}.

Na prática, dentro do processo de desenvolvimento de software livre, após lançar
uma versão inicial e divulgar o projeto, os usuários interessados começam a
usar o software livre em questão. 
%
De acordo com Eric Raymond,``bons programas nascerem de
necessidades pessoais'', esses usuários podem também ser desenvolvedores, que
irão colaborar com o projeto a fim de atenderem às suas próprias necessidades.
%
Destacando a colaboração no código-fonte, essas melhorias são enviadas aos
mantenedores do projeto como \emph{patches}, ou seja, arquivos que
contém as modificações no código e que serão analisados pelos mantenedores que,
caso concordem com a mudança e com a sua implementação em si, irão
aplicá-las ao repositório oficial do projeto.
%
Portanto, mesmo que em projetos maiores outros aspectos sejam levados em consideração ou
sigam processos mais burocrático de colaboração, a essência da colaboração
técnica está no envio e análise de trechos de código-fonte \cite{meirelles2013metrics}.


\section{Padrões de Software Livre}
\label{sec-padroes-sl} 

Um software livre é concebido através de um processo de contribuições, o qual possui características especiais que promovem o surgimento de diversas práticas influenciadas por diversas forças. Tais práticas são conhecidas como padrões de software livre. Para simplificar, nesta seção o termo padrão está associado a padrões de software livre. Esses padrões estão organizados dentro de três grupos:

\begin{itemize}

	\item \textbf{Padrões de seleção} auxiliam prováveis colaboradores a selecionar projetos adequados.

		\begin{itemize}

			\item O primeiro padrão de seleção recomenda colaboradores novatos a "caminhar sobre terreno conhecido", ou seja, se deseja contribuir, começar por algum software que seja familiar, como por exemplo, um browser, editor de texto, IDE\footnote{Integrated Development Environment, um ambiente integrado para desenvolvimento de software}, ou qualquer outro software que já se utiliza.

			\item O segundo padrão é similar ao primeiro, porém ao invés da ferramenta ser familiar, esse padrão recomenda que o colaborador tenha conhecimentos na linguagem ou tecnologia utilizada no projeto.

			\item Já o terceiro padrão desse grupo motiva colaboradores a procurar por projetos de software livre que ofereçam fucionalidades atrativas, mesmo que o novo colaborador não tenha familiaridade com a ferramenta nem com a tecnologia utilizada em seu desenvolvimento.
		\end{itemize}

O terceiro padrão é o que melhor se encaixa ao contexto desse trabalho, já que o Mezuro não era uma ferramenta utilizada no cotidiano e tão pouco familiar. Além disso, as tecnologias utilizadas em seu desenvolvimento não eram as de maior conhecimento. Entretanto, as funcionalidades providas por essa plataforma foi determinante para essa contribuição.

	\item \textbf{Padrões de envolvimento} lidam com os primeiros passos para que o colaborador se familiarize e se envolva com o projeto selecionado.

		\begin{itemize}
			\item Entrar em contato com mantenedores para aprender sobre o contexto 	histórico e político no qual aquele projeto está inserido.

			\item Realizar instalação e checar se todo o ambiente do projeto está 	corretamente configurado em um período limitado de tempo (máximo um dia)

			\item Durante uma apresentação do sistema,  por parte de algum mantenedor, interagir para se familiarizar melhor com funcionalidades e cenários presentes no sistema.

			\item Avaliar o estado do sistema através de uma breve, mas intensa revisão de código. Isso ajuda a ter uma primeira impressão sobre a 					qualidade do código-fonte.

			\item Através da leitura, avaliar a relevancia da documentação em um 			período limitado de tempo.

			\item Checar a lista de tarefas a serem feitas. Ela pode conter bons 			pontos de partida para começar uma contribuição.

			\item Relacionado ao padrão mencionado acima, está o padrão que recomenda novos colaboradores iniciarem por tarefas mais fáceis. Começar uma tarefa 			e termina-la é importante para manter colaboradores motivados, e conforme 			ganharem mais experiencia e familiaridade com o software avançam para 			tarefas mais complexas.
	\end{itemize}

No contexto deste trabalho, muitos dos padrões desse grupo foram inseridos ao processo de contribuição. Por exemplo, o orientador deste trabalho é também mantenedor da plataforma Mezuro, assim como outros colaboradores da plataforma, auxiliaram durante o processo de envolvimento, apresentando funcionalidades e principais cenários do sistema, além de fornecer documentação necessaŕia para o entendimento do histórico e contexto no qual o Mezuro está inserido.
	
	\item \textbf{Padrões de contribuição} documenta as melhores práticas para 	se contribuir com softwares livres. Os grupos anteriores tratavam como iniciar 	e se familiarizar com um projeto de software livre. Esse grupo, por sua vez, 		contém padrões que auxiliam o fornecimento de insumos para projetos de software livre, seja codigo-fonte ou outros artefatos presentes no processo de desenvolvimento.

	\begin{itemize}

		\item Uma boa contribuição para projetos de software em geral, é a escrita de documentação. O código-fonte muitas vezes não é o suficiente para que todos os envolvidos entendam o andamento do projeto, pois apesar de promoverem o software não possuem conhecimento técnico suficiente. Além disso, documentação do projeto auxilia na manutenção e evolução do produto.

		\item Muitos softwares livres não suportam o idioma de diversos colaboradores. Um bom ponto de partida seria a internacionalização do sistema, incluindo a própria linguagem no sistema.

		\item Reportar bugs eficientemente, pois é comum que colaboradores identifiquem bugs mas ao reporta-los não são claros com respeito ao seu contexto, dificultado sua correção.

		\item Utilizar a versão correta para tarefas. Durante o desenvolvimento de software há diferentes versões, onde há no mínimo uma versão estável e uma versão desenvolvimento. É recomendado utiliar a versão estável para reportar bugs e a versão de desenvolvimento para implementar novas funcionalidades e tudo que não está relacionado com correção de defeitos existentes.

		\item Separar alterações não relacionadas. Se tratando de sistemas de controle de versão\footnote{SCM - Source Code Management - GIT, SVN, Baazar, Mercurial, entre outros} há uma ação conhecida como commit, onde as alterações realizadas são agrupadas e gravadas. É recomendado que num mesmo commit as alterações sejam relacionadas.

		\item Mensagens de commit explicativas para facilitar o entendimento e identificação do que foi desenvolvido ou alterado para o restante dos colaboradores.

		\item Documentar as próprias modificações. Desenvolvedores, geralmente, alteram o código, corrigem bugs, adicionam novas funcioanlidades, mas não atualizam a documentação, a qual se torna desatualizada. Por isso é recomendado documentar as alterações antes de submete-as ao repositório.

		\item Manter-se atualizado com o estado atual do projeto, ajudando a evitar duplicação de esforços e identificar oportunidades de colaboração. Isso é importante pois um projeto de software livre é um esforço coletivo, mas às vezes é difícil coordenar o esforço de pessoas com diferentes horários e prioridades.

	\end{itemize}

\end{itemize}

%TODO: Faltou um sincronização deste padrões com o seu trabalho.

Em resumo, esses padrões não são regras, apenas indicam um bom caminho para contribuições. Por exemplo, o software tratado neste trabalho, o autor principal do mesmo, não era familiar no início do processo de contribuição para a colaboração da evolução de um software livre.

%TODO: o capítulo está bem incompleto e não conclusivo. Deve melhora
